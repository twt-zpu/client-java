% This file was converted to LaTeX by Writer2LaTeX ver. 1.4
% see http://writer2latex.sourceforge.net for more info
\documentclass{article}
\usepackage[latin1]{inputenc}
\usepackage[T3,T1]{fontenc}
\usepackage[swedish]{babel}
\usepackage[noenc]{tipa}
\usepackage{tipx}
\usepackage[geometry,weather,misc,clock]{ifsym}
\usepackage{pifont}
\usepackage{eurosym}
\usepackage{amsmath}
\usepackage{wasysym}
\usepackage{amssymb,amsfonts,textcomp}
\usepackage{array}
\usepackage{supertabular}
\usepackage{hhline}
\usepackage[pdftex]{graphicx}
\makeatletter
\newcommand\arraybslash{\let\\\@arraycr}
\makeatother
\setlength\tabcolsep{1mm}
\renewcommand\arraystretch{1.3}
\newcounter{Table}
\renewcommand\theTable{\arabic{Table}}

\usepackage{lastpage}
\usepackage[hmargin=3cm,top=6cm,headheight=5cm,footskip=65pt]{geometry}

\usepackage{fancyhdr}
\fancyhf{}
\lhead{
\includegraphics[width=2cm]{fig/Arrowhead_logo}
}
\rhead{%
  \begin{tabular}{|p{8cm}|p{3cm}|}
\hline
    \small{Document title} & \small{Document type} \\
    SysDD Temaplate - White box design & Template \\
    \small{Date} & \small{Version} \\
    \date{\today} & 1.2 \\
    \small{Author} & \small{Status} \\
    Fredrik Blomstedt & Proposed \\
    \small{Contact} & \small{Page} \\
    fredrik.blomstedt@bnearit.se & \thepage (\pageref{LastPage})\\ \hline
  \end{tabular}%
}

\lfoot{
\includegraphics[width=2cm]{fig/Artemis_logo}
}
\rfoot{Contributed by the Arrowhead project, www.arrowhead.eu}

\renewcommand*{\headrulewidth}{0pt}
\pagestyle{fancy}


\title{Service Description (SD) Template}
\begin{document}
\maketitle
\setcounter{tocdepth}{10}
\renewcommand\contentsname{}
\tableofcontents
\newpage

\section[Service Description Overview]{Service Description Overview}
This document describes an Arrowhead service, including its interfaces, functions and information model. Some UML diagrams describing an abstract architecture of the service should be provided in this section. 

The type of service that will be described should also be defined. The service type identifier clearly depicts the service that is going to be provided (e.g. Temperature). 

%An example of how this document can be elaborated can be found in SVN repository {\dots}/Meetings/Multi WP Workshops/2013-11-05 Porto/Documenting/Examples/SD/Arrowhead SD Temperature. 

\section[Abstract Interfaces]{Abstract Interfaces}
Every interface should be described in a separate section. A more detailed description of the Overview section diagrams is suggested. 

Every function included in the interface should also be presented and explained in a subsection for each interface. Sequence diagrams might be included to give a more clear view on functions' usage. 

The use of UML or SysML diagrams can be an adequate solution.

\subsection[Interface 1]{Interface 1}
\subsubsection{Functions}
\paragraph{Funtion 1 from Interface 1}
\paragraph{Funtion 2 from Interface 1}
\subsubsection{Sequence Diagrams}
\paragraph[Sequence Diagram 1 ]{Sequence Diagram 1 }
\paragraph[Sequence Diagram 2]{Sequence Diagram 2}
\subsection{Interface 2}
...

\section[Abstract Information Model]{Abstract Information Model}
Give a high level presentation of the information model with types, attributes and relationships. 

Include if possible one of the following diagrams:

\begin{itemize}
\item UML Class diagram
\item SysML Parametric diagram
\end{itemize}
Explain the each data typefield, its attributes and any relations to other typesfields. A suggested approach is depicted in Table 1.

Table \stepcounter{Table}{\theTable} Data type description

\begin{flushleft}
\tablefirsthead{}
\tablehead{}
\tabletail{}
\tablelasttail{}
\begin{supertabular}{|m{1.0650599in}|m{4.8677597in}|}
\hline
Field &
Description\\\hline
 &
\\\hline
\end{supertabular}
\end{flushleft}
Not mandatory but also possible, in this section to provide any metadata defined by the service. Metadata is typically information that describes an instance of the service, like position, measuring unit etc. 

Table \stepcounter{Table}{\theTable} Metadata description

\begin{flushleft}
\tablefirsthead{}
\tablehead{}
\tabletail{}
\tablelasttail{}
\begin{supertabular}{|m{1.0483599in}|m{3.5490599in}|m{1.2573599in}|}
\hline
Tag &
Description &
Mandatory\\\hline
 &
 &
\\\hline
\end{supertabular}
\end{flushleft}
\section[Non{}-functional Requirements]{Non-functional Requirements}
Describe requirements regarding QoS, response time, resources, reliability, etc. This section lists the non-functional requirements that can be implemented by the service. Note that some of these requirements can be optional.

This section reports, for example, the reliability targets that must be respected by the service in a particular use case. For example, a service can offer acknowledgment to service requests in a use case, and act on a best effort base in another use case.

The following table reports multiple non-functional requirements sets, each one pertaining to a use case that are directly implemented by the service.

Table \stepcounter{Table}{\theTable} Non-functional requirements description

\begin{flushleft}
\tablefirsthead{}
\tablehead{}
\tabletail{}
\tablelasttail{}
\begin{supertabular}{|m{0.8997598in}|m{0.9691598in}|m{0.87755984in}|m{1.4955599in}|m{1.3608599in}|}
\hline
Name &
Description &
Type &
Constraints &
Use{}-case\\\hline
 &
 &
 &
 &
\\\hline
\end{supertabular}
\end{flushleft}
The above table must report the non-functional requirements, by providing information such as:

Name: The unique name (UID) of the non-functional requirement.

Description: A description of the non-functional requirement.

Type: The type of the non-functional requirement (e.g., hardware, software, performance).

Service: The Name or ID of the service that these non-functional requirements are referring to.

Constraints: Any constraints the requirement imposes to the use-case(s) (e.g., have a delay of less than 1 ms to deliver the service).

Use-case: Provide the ID of the use case(s) it refers to.

\section[\ \ References]{\ \ References}
Any references must be placed here

\section[Revision history]{Revision history}
\subsection[Amendments]{Amendments}
\begin{flushleft}
\tablefirsthead{}
\tablehead{}
\tabletail{}
\tablelasttail{}
\begin{supertabular}{|m{0.38445985in}|m{1.1587598in}|m{0.5545598in}|m{2.07476in}|m{1.5247599in}|}
\hline
No. &
Date &
Version &
Subject of Amendments &
Author\\\hline
1 &
2013-12-18 &
0.3 &
Text Revisions &
Christos Chrysoulas\\\hline
2 &
2014-01-11 &
0.4 &
Quickparts \& template &
Ove Jansson\\\hline
3 &
2015-02-20 &
1.0 &
Text revision &
Michele Albano / Luis Ferreira\\\hline
4 &
2015-09-30 &
1.1 &
Refinement of the structure &
Michele Albano / Luis Ferreira\\\hline
5 & 2016-03-19 & 1.2 & Transfer to Latex & Jerker Delsing \\ \hline

\end{supertabular}
\end{flushleft}
\subsection[Quality Assurance]{Quality Assurance}
\begin{flushleft}
\tablefirsthead{}
\tablehead{}
\tabletail{}
\tablelasttail{}
\begin{supertabular}{|m{0.38865986in}|m{1.2011598in}|m{0.5559598in}|m{1.5656599in}|}
\hline
No. &
Date &
Version &
Approved by\\\hline
1 &
 &
 &
\\\hline
2 &
 &
 &
\\\hline
\end{supertabular}
\end{flushleft}
\end{document}
