% This file was converted to LaTeX by Writer2LaTeX ver. 1.4
% see http://writer2latex.sourceforge.net for more info
\documentclass{article}
\usepackage[latin1]{inputenc}
\usepackage[T3,T1]{fontenc}
\usepackage[english]{babel}
\usepackage[noenc]{tipa}
\usepackage{tipx}
\usepackage[geometry,weather,misc,clock]{ifsym}
%\usepackage{pifont}
\usepackage{eurosym}
\usepackage{amsmath}
\usepackage{wasysym}
\usepackage{amssymb,amsfonts,textcomp}
\usepackage{array}
\usepackage{supertabular}
\usepackage{hhline}
\usepackage[pdftex]{graphicx}
\usepackage{lastpage}
\usepackage[hmargin=3cm,top=5cm,headheight=4cm,footskip=65pt]{geometry}

\usepackage{fancyhdr}
\fancyhf{}
\lhead{
\includegraphics[width=2cm]{fig/Arrowhead_logo}
}
\rhead{%
  \begin{tabular}{|p{8cm}|p{3cm}|}
\hline
    \small{Document title} & \small{Document type} \\
    SysDD Temaplate - White box design & Template \\
    Date & Version \\
    \date{\today} & 1.2 \\
    Author & Status \\
    Fredrik Blomstedt & Proposed \\
    Contact & Page \\
    fredrik.blomstedt@bnearit.se & \thepage (\pageref{LastPage})\\ \hline
  \end{tabular}%
}

\lfoot{
\includegraphics[width=2cm]{fig/Artemis_logo}
}
\rfoot{Contributed by the Arrowhead project, www.arrowhead.eu}

\renewcommand*{\headrulewidth}{0pt}
\pagestyle{fancy}

\newcounter{Table}
\renewcommand\theTable{\arabic{Table}}


\title{SysDD Temaplate - White box design}
%\author{Jerker Delsing, jerker.delsing@ltu.se}
\date{\empty}

\begin{document}

\maketitle


%\section{Table of Contents}
\setcounter{tocdepth}{10}
%\renewcommand\contentsname{}

\tableofcontents

\newpage

This document is optional. We are here recommending a structure for
this document, but the list of sections can be modified on a case by
case basis. It can include one or more eye catching figures into this description.

This document does not report non-functional requirements (QoS,
robustness, etc) since they are related to the interaction between
systems and to the SoS as a whole, and are thus defined in the SoSD
(objectives) and SoSDD documents (implementations).

\section{System Design Description Overview}

In Table \ref{tab:sys_info} the global System Information is given.

\begin{table}[h!]
  \centering
  \begin{tabular}{|p{5cm}|p{8cm}|}
\hline
    Name & An unique name readable by humans \\ \hline
    Maintainer  & e.g. Jerker Delsing, jerker.delsing@ltu.se \\ \hline 
    SySD document & Path in Arrowhead Framework repository \\ \hline
  \end{tabular}
  \caption{System information}
  \label{tab:sys_info}
\end{table}

It can include one or more eye catching figures into this description.

This document is optional. We are here recommending a structure for this document, but the list of sections can be modified on a case by case basis.

This document does not report non-functional requirements (QoS, robustness, etc) since they are related to the interaction between systems and to the SoS as a whole, and are thus defined in the SoSD (objectives) and SoSDD documents (implementations).

%An example of how this template can be used can be found on the SVN server in: {\dots}{\dots}{\dots}.Arrowhead{\textbackslash}Meetings{\textbackslash}Multi WP Workshops{\textbackslash}2013-11-05 Porto{\textbackslash}Documenting{\textbackslash}Examples{\textbackslash}SysDD{\textbackslash} Arrowhead SysDD Prosumer\_v0.1.docx.

\section{Use-cases}
Describe typical use cases e.g. using UML use cases diagrams, which can be realised by the system.

Only add this section if relevant in relation to SysD document. Each use-case description should follow the structure defined in Table 3.

Table \stepcounter{Table}{\theTable} Use-case description

\begin{flushleft}
\tablefirsthead{}
\tablehead{}
\tabletail{}
\tablelasttail{}
\begin{supertabular}{|m{6.16916in}|}
\hline
Name of the Use-case\\\hline
ID: Unique ID\\\hline
Brief description:

Give a brief description of the use-case.\\\hline
Primary actors:

Present the primary actor, e.g.,Prosumer\\\hline
Secondary actors:

Present the secondary actors, e.g., Virtual Market of Energy\\\hline
Preconditions:

If there are any\\\hline
Main flow:

Present in a sequence of steps the interactions among the actors 

1-

2-

3-

{\dots}{\dots}{\dots}\\\hline
Postconditions:

If there are any\\\hline
Alternative flows:

Any possible alternative flows to the sequence presented in the Main flow section.\\\hline
\end{supertabular}
\end{flushleft}
\section{Typical Scenarios (optional)}
Table 4 Typical Scenario description

\begin{flushleft}
\tablefirsthead{}
\tablehead{}
\tabletail{}
\tablelasttail{}
\begin{supertabular}{|m{1.4726598in}|m{4.5643597in}|}
\hline
{\selectlanguage{english} Typical Scenario} &
{\selectlanguage{english} For white-box design, describe typical interactions between required and provided service e.g. }

\begin{itemize}
\item {\selectlanguage{english} using UML sequence diagrams, which can be used as test cases for system integration testing}
\item {\selectlanguage{english} implementing UML or SysML activity diagrams, which can be used as process definition for the system or the integration between systems}
\end{itemize}
\\\hline
\end{supertabular}
\end{flushleft}
\section[Internal structure]{Internal structure}
This section reports the internal structure of the system in terms of classes, and the location of the classes into the program files.

The UML Component diagram or the SysML Block Definition diagram are recommended for a formal description of the class structure.

Pointers to the location of the program files can be given using a table.

Table 4 Program files

\begin{flushleft}
\tablefirsthead{}
\tablehead{}
\tabletail{}
\tablelasttail{}
\begin{supertabular}{|m{1.4726598in}|m{4.5643597in}|}
\hline
{\selectlanguage{english} Class name} &
{\selectlanguage{english} Name of the file(s) / optional location of the file(s)}\\\hline
\end{supertabular}
\end{flushleft}
\section[Security ]{Security }
This chapter describes how security is implemented into the systems.

\subsection[Technical Security Requirements]{Technical Security Requirements}
In this chapter the technical realization of the non-technical System security requirements, specified in chapter 5.3 of the document SySD is specified.

\subsection{Decomposition of the System}
In this section a detailed description of the System implementation is required.  The description needs to cover the following points:

\begin{itemize}
\item Architecture of the System (diagrams, explanations, technologies used)
\item Technical description of interface implementation (interfaces between Systems)
\item Description of implementation of access control mechanisms
\item Programming languages used
\end{itemize}
\subsection{Technical Security Requirements}
In this chapter the technical realization of the non-technical System security requirements, specified in chapter 5.3 of the document SySD is specified.

\subsection{Data Flow Diagram}
Provide a Data Flow diagram of the System.

\subsection{Threats and Vulnerabilities}
This chapter contains any known threats and vulnerabilities to the System.

\section[References]{References}
Any references must be placed here

\section[Revision history]{Revision history}
\subsection[Amendments]{Amendments}
\begin{flushleft}
\tablefirsthead{}
\tablehead{}
\tabletail{}
\tablelasttail{}
\begin{supertabular}{|m{0.38935986in}|m{1.2011598in}|m{0.5545598in}|m{2.1420598in}|m{1.5670599in}|}
\hline
{\selectlanguage{english} No.} &
{\selectlanguage{english} Date} &
{\selectlanguage{english} Version} &
{\selectlanguage{english} Subject of Amendments} &
{\selectlanguage{english} Author}\\\hline
{\selectlanguage{english} 1} &
{\selectlanguage{english} 2013-12-03} &
{\selectlanguage{english} 0.2} &
{\selectlanguage{english} Revision} &
{\selectlanguage{english} Christos Chrysoulas}\\\hline
{\selectlanguage{english} 2} &
{\selectlanguage{english} 2013-12-18} &
{\selectlanguage{english} 0.3} &
{\selectlanguage{english} Text Revisions} &
{\selectlanguage{english} Christos Chrysoulas}\\\hline
{\selectlanguage{english} 3} &
{\selectlanguage{english} 2013-12-24} &
{\selectlanguage{english} 0.4} &
{\selectlanguage{english} Added quick parts and saved as template} &
{\selectlanguage{english} Ove Jansson}\\\hline
{\selectlanguage{english} 4} &
{\selectlanguage{english} 2015-02-19} &
{\selectlanguage{english} 1.0} &
{\selectlanguage{english} Revision} &
{\selectlanguage{english} Michele Albano / Luis Ferreira}\\\hline
{\selectlanguage{english} 7} &
{\selectlanguage{english} 2015-09-30} &
{\selectlanguage{english} 1.1} &
{\selectlanguage{english} Refinement of the structure} &
{\selectlanguage{english} Michele Albano / Luis Ferreira}\\\hline
{\selectlanguage{english} 8} &
{\selectlanguage{english} 2015-09-30} &
{\selectlanguage{english} 1.1} &
{\selectlanguage{english} Revision} &
{\selectlanguage{english} Iker Mart�nez de Soria}\\\hline
9 & 2016-03-19 & 1.2 & Transfer to Latex & Jerker Delsing \\ \hline
\end{supertabular}
\end{flushleft}
\subsection[Quality Assurance]{Quality Assurance}
\begin{flushleft}
\tablefirsthead{}
\tablehead{}
\tabletail{}
\tablelasttail{}
\begin{supertabular}{|m{0.38865986in}|m{1.2011598in}|m{0.5559598in}|m{1.5656599in}|}
\hline
{\selectlanguage{english} No.} &
{\selectlanguage{english} Date} &
{\selectlanguage{english} Version} &
{\selectlanguage{english} Approved by}\\\hline
{\selectlanguage{english} 1} &
 &
 &
\\\hline
{\selectlanguage{english} 2} &
 &
 &
\\\hline
\end{supertabular}
\end{flushleft}
\end{document}
