% This file was converted to LaTeX by Writer2LaTeX ver. 1.4
% see http://writer2latex.sourceforge.net for more info
\documentclass{article}
\usepackage[latin1]{inputenc}
\usepackage[T3,T1]{fontenc}
\usepackage[english,swedish]{babel}
\usepackage[noenc]{tipa}
\usepackage{tipx}
\usepackage[geometry,weather,misc,clock]{ifsym}
\usepackage{pifont}
\usepackage{eurosym}
\usepackage{amsmath}
\usepackage{wasysym}
\usepackage{amssymb,amsfonts,textcomp}
\usepackage{array}
\usepackage{supertabular}
\usepackage{hhline}
\usepackage[pdftex]{graphicx}
\makeatletter
\newcommand\arraybslash{\let\\\@arraycr}
\makeatother
\setlength\tabcolsep{1mm}
\renewcommand\arraystretch{1.3}
\newcounter{Table}
\renewcommand\theTable{\arabic{Table}}

\usepackage{lastpage}
\usepackage[hmargin=3cm,top=6cm,headheight=5cm,footskip=65pt]{geometry}

\usepackage{fancyhdr}
\fancyhf{}
\lhead{
\includegraphics[width=2cm]{fig/Arrowhead_logo}
}
\rhead{%
  \begin{tabular}{|p{8cm}|p{3cm}|}
\hline
    \small{Document title} & \small{Document type} \\
    SysDD Temaplate - White box design & Template \\
    \small{Date} & \small{Version} \\
    \date{\today} & 1.2 \\
    \small{Author} & \small{Status} \\
    Fredrik Blomstedt & Proposed \\
    \small{Contact} & \small{Page} \\
    fredrik.blomstedt@bnearit.se & \thepage (\pageref{LastPage})\\ \hline
  \end{tabular}%
}

\lfoot{
\includegraphics[width=2cm]{fig/Artemis_logo}
}
\rfoot{Contributed by the Arrowhead project, www.arrowhead.eu}

\renewcommand*{\headrulewidth}{0pt}
\pagestyle{fancy}


\title{Communication Profile (CP) Template}
\begin{document}
\maketitle
\tableofcontents
\newpage

\section[Communication Profile Overview]{Communication Profile Overview}
In this section a complete description of technologies and specifications that build the Communication Profile must be provided. A Communication Profile Identifier must be the outcome of the present document. The Communication Profile Identifier should be described in a subsection. The identifier is composed by the Transfer protocol (e.g. CoAP), by the Security that will be used for the transportation of data (e.g. DTLS), and by the data format (e.g. EXI). Therefore, an example of an identifier can be the following: CoAP-DTLS-EXI.

%An example of how this document can be elaborated can be found in SVN repository {\dots}{\textbackslash}Meetings{\textbackslash}Multi WP Workshops{\textbackslash}2013-11-05 Porto{\textbackslash}Documenting{\textbackslash}Examples{\textbackslash}IDD{\textbackslash}CP{\textbackslash}Arrowhead CP CoAP-DTLS-EXI.

\section[Message Exchange Patterns / Protocols]{Message Exchange Patterns / Protocols}
This chapter explains how to implement a common set of message exchange patterns using this communication profile.

Every message exchange patterns must be described in detail in its sub-sections. Examples of Message Exchange Patterns can be: Request-Response, Publish-Subscribe, One-to-Many, etc.

For each message exchange pattern, a number of Protocols will be described. These ones are are the the list of transfer protocols employed in the communication profile, such as CoAP.

Later in this section, the protocols are instantiated to be used to interact with the resources on the communication profile. For example, this section can report how the REST operations are mapped over the functions.

This section describes how the capabilities related to the Communication Profile are used.

For instance, iIn a CoAP- based exampleapplication the functions used to encode and transfer data using this communication profile, thus according to CoAP, should be documented like in Table 2, in a subsection. Adaptation of this section are allowed to fit the specificities of the use case.:

Table \stepcounter{Table}{\theTable} Function description

\begin{flushleft}
\tablefirsthead{}
\tablehead{}
\tabletail{}
\tablelasttail{}
\begin{supertabular}{|m{1.1448599in}|m{1.1393598in}|m{1.1413599in}|m{1.1316599in}|m{1.1400598in}|}
\hline
{\selectlanguage{english} Function} &
{\selectlanguage{english} Service} &
{\selectlanguage{english} Method} &
{\selectlanguage{english} Input} &
{\selectlanguage{english} Output}\\\hline
 &
 &
 &
 &
\\\hline
\end{supertabular}
\end{flushleft}
Another subsection regarding data, either input or output should be described using XML Schemas.See the examples in the repository.

\section[Security]{Security}
Define in detail how the proposed Communication Profile handles security issues, regarding Authentication, based on the protocol specifications. For instance in the use of CoAP, DTLS is enabled.

\section[Endpoint Description]{Endpoint Description}
For instance, in a CoAP based example, an endpoint utilizing this communication profile must expose the following information to its communicating parties.

Table \stepcounter{Table}{\theTable} Endpoint characteristics -- IP based protocol

\begin{flushleft}
\tablefirsthead{}
\tablehead{}
\tabletail{}
\tablelasttail{}
\begin{supertabular}{|m{1.3566599in}|m{4.57616in}|}
\hline
{\selectlanguage{english} IP} &
{\selectlanguage{english} IP address}\\\hline
{\selectlanguage{english} Port} &
{\selectlanguage{english} UDP port}\\\hline
{\selectlanguage{english} Service} &
{\selectlanguage{english} Resource identifier }\\\hline
\end{supertabular}
\end{flushleft}
Table 2 Endpoint characteristics - XMPP

\begin{flushleft}
\tablefirsthead{}
\tablehead{}
\tabletail{}
\tablelasttail{}
\begin{supertabular}{|m{1.3566599in}|m{4.57616in}|}
\hline
{\selectlanguage{english} Server URL} &
{\selectlanguage{english} URL of the XMPP server}\\\hline
{\selectlanguage{english} Chat Room} &
{\selectlanguage{english} XMPP Chat room id}\\\hline
\end{supertabular}
\end{flushleft}
A valid URL should be provided to allow connection to the service.

Note that this table must be adapted for the specificities of the use case. In XMPP, these is no IP/Port for the endpoint, but instead chat rooms. In AMQP there are exchanges and queues.

\section[Metadata]{Metadata}
If any metadata is available should be presented in this section.

\section[Data Encoding ]{Data Encoding }
Describe the data format used by the communication profile and, if possible, provide an example of a message. Examples could be XML, EXI, JSON etc.

\section[Description Format]{Description Format}
This section describes how the capabilities related to the Communication Profile are used.

For instance, in a CoAP based example the functions used to encode and transfer data using this communication profile, thus according to CoAP, should be documented like in Table 2, in a subsection:

Table \stepcounter{Table}{\theTable} Function description

\begin{flushleft}
\tablefirsthead{}
\tablehead{}
\tabletail{}
\tablelasttail{}
\begin{supertabular}{|m{1.1448599in}|m{1.1393598in}|m{1.1413599in}|m{1.1316599in}|m{1.1400598in}|}
\hline
{\selectlanguage{english} Function} &
{\selectlanguage{english} Service} &
{\selectlanguage{english} Method} &
{\selectlanguage{english} Input} &
{\selectlanguage{english} Output}\\\hline
 &
 &
 &
 &
\\\hline
\end{supertabular}
\end{flushleft}
Another subsection regarding data, either input or output should be described using XML Schemas.

\section[Standards and Demarcations]{Standards and Demarcations}
Present a list of specifications used by the Communication Profile, in the following format: 

Table \stepcounter{Table}{\theTable} Specifications list

\begin{flushleft}
\tablefirsthead{}
\tablehead{}
\tabletail{}
\tablelasttail{}
\begin{supertabular}{|m{3.1538599in}|m{1.8323599in}|m{0.8677598in}|}
\hline
{\selectlanguage{english} Specification} &
{\selectlanguage{english} Type} &
{\selectlanguage{english} Version}\\\hline
 &
 &
\\\hline
 &
 &
\\\hline
 &
 &
\\\hline
\end{supertabular}
\end{flushleft}
Any demarcations in the usage of standards should be presented in this section. 

\section[References]{References}
Any references must be placed here

\section[Revision history]{Revision history}
\subsection[Amendments]{Amendments}
\begin{flushleft}
\tablefirsthead{}
\tablehead{}
\tabletail{}
\tablelasttail{}
\begin{supertabular}{|m{0.38305986in}|m{1.1441599in}|m{0.6219598in}|m{2.05316in}|m{1.4948599in}|}
\hline
{\selectlanguage{english} No.} &
{\selectlanguage{english} Date} &
{\selectlanguage{english} Version} &
{\selectlanguage{english} Subject of Amendments} &
{\selectlanguage{english} Author}\\\hline
{\selectlanguage{english} 1} &
{\selectlanguage{english} 2015-02-20} &
{\selectlanguage{english} 1.0} &
{\selectlanguage{english} Revision of text} &
{\selectlanguage{english} Michele Albano / Luis Ferreira}\\\hline
{\selectlanguage{english} 2} &
{\selectlanguage{english} 2015-09-3018} &
{\selectlanguage{english} 1.1} &
{\selectlanguage{english} Refinement of the textstructure} &
{\selectlanguage{english} Michele Albano / Luis Ferreira}\\\hline
3 & 2016-03-19 & 1.2 & Transfer to Latex & Jerker Delsing \\ \hline

\end{supertabular}
\end{flushleft}
\subsection[Quality Assurance]{Quality Assurance}
\begin{flushleft}
\tablefirsthead{}
\tablehead{}
\tabletail{}
\tablelasttail{}
\begin{supertabular}{|m{0.38865986in}|m{1.2011598in}|m{0.6219598in}|m{1.5656599in}|}
\hline
{\selectlanguage{english} No.} &
{\selectlanguage{english} Date} &
{\selectlanguage{english} Version} &
{\selectlanguage{english} Approved by}\\\hline
{\selectlanguage{english} 1} &
 &
 &
\\\hline
{\selectlanguage{english} 2} &
 &
 &
\\\hline
\end{supertabular}
\end{flushleft}
\end{document}
